\appendix

\chapter{Excess Interest}

The `par' rate is the true mortgage interest rate available to all consumers based on their individual default risk. It is available when consumers pay the broker/lender sales commissions directly in cash. Since this would highlight how large the commissions are (inviting negotiation) and might reduce consumer purchasing power by cannibalizing from the downpayment, broker/lenders almost never make consumers aware of the par rate (except for friends and family). Instead, broker/lenders choose to have their sales commissions paid by the loan funding source.

Since the funding source must provide \(>100\%\) of the loan value to cover the sales commissions but by law the loan principal amount cannot be increased, mechanically the interest rate is the only variable available to change within the funding equation.

Empirically, the interest rate increases at a 1:2 ratio, rising by 150 bps on average to accommodate 300 bps of sales commissions. This rate increase applies to the entire loan amount over the full mortgage term, inflating lifetime interest costs far in excess of the sales commission amount.

We term ``excess interest'' as the cumulative increase in interest paid over and above the value of the sales commissions themselves. Excess interest therefore precisely quantifies the tax on the consumer due to embedding the sales commissions in the interest rate (``rate-embedding'') vs. the consumer paying the sales commissions directly in cash.

Rate-embedded commissions turn a one-time sales fee into a lifetime tax that grows in proportion to the total hold period. Excess interest totals 7\% and 30\% of the home's value over a 7 and 30 year hold period respectively. For most consumers, excess interest is the single largest transaction cost. This enormous tax is entirely eliminated on Elephant Protocol.

Industry insiders do not talk about excess interest for two reasons: i) sales people handle commissions and commission mechanisms and do not understand the amortization math or ii) they view it as an unfortunate but necessary mechanism to help consumers maximize their purchasing power. Of course, the former is inexcusable and the latter fails to recognize that at today's rate a 150 bps lower mortgage rate increases purchasing power by 12\%, all else equal, far in excess of the reduction in downpayment due to paying sales commissions in cash.