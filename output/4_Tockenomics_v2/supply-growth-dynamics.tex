\section{Supply Growth Dynamics}

MAHOUT's supply growth is intrinsically tied to real-world asset verification rather than arbitrary emission schedules. As more properties are added to the network and verified by oracles, new tokens are minted to reward the verification work. This creates a natural correlation between token supply and network utility—more tokens exist only when more verified property data exists to support them.

The total potential token supply is effectively capped by the number of properties in existence multiplied by the fact groups per property. With approximately 140 million properties in the United States and 20 fact groups per property, the maximum theoretical supply would be 2.8 billion MAHOUT tokens. However, practical supply will be much lower as properties are gradually onboarded and some fact groups may not require updates for extended periods.

This supply model creates unique economic dynamics. Early oracles can accumulate tokens more easily as unverified properties are plentiful. As the network matures and most properties are verified, new token emission slows naturally, creating scarcity that rewards early contributors while maintaining incentives for ongoing maintenance.

