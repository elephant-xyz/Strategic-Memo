\section{Truth Mining}

At the core of Elephant's economic model lies Truth Mining, a consensus mechanism that transforms property data verification into a sustainable economic activity. The protocol divides property information into 20 independent fact groups, each requiring verification from three unique oracles. This granular approach ensures that no single entity can monopolize the truth verification process while maintaining data integrity through mathematical consensus.

When oracles submit data for verification, they stake their reputation and tokens on the accuracy of their contributions. The protocol requires unanimous consensus among all three oracles before any data is permanently written to the blockchain and rewards are distributed. This creates a powerful economic incentive for accuracy—oracles who submit false or careless data not only forfeit their potential rewards but also risk losing their staked tokens.

The reward distribution follows an exponential curve designed to incentivize rapid, accurate submissions. For each fact group verified, exactly 1 MAHOUT token is minted and distributed among the three oracles. The first oracle to submit verified data receives approximately 81\% of the reward, the second oracle receives 16\%, and the third oracle receives 3\%. This exponential reward structure creates urgency in the marketplace while still ensuring multiple independent verifications of each data point.

\[
\text{Reward Distribution} = \begin{cases}
\text{First Oracle: } 0.81 \text{ MAHOUT} \\
\text{Second Oracle: } 0.16 \text{ MAHOUT} \\
\text{Third Oracle: } 0.03 \text{ MAHOUT}
\end{cases}
\]

