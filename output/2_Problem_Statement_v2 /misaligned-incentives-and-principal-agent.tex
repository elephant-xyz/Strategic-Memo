\section{Misaligned Incentives and Principal-Agent}

Service providers in real estate have perfected the art of necessary inefficiency. Their incentives align not toward streamlining transactions but toward ensuring their permanent involvement. Why create systems that remember when forgetting generates \$8,600 per transaction? Why enable direct communication when coordination failures justify commission? Why reduce friction when friction is the product?

This misalignment manifests most perniciously in pricing structures tied to asset values rather than actual work. Agent fees scale with home prices—a \$2 million property pays ten times the commission of a \$200,000 home despite requiring identical effort. This percentage-based extraction creates powerful incentives to inflate asset prices, driving the very unaffordability that locks millions out of homeownership. The system profits from scarcity, complexity, and inflated values—precisely the opposite of what society needs.

Reputation tracking remains deliberately absent, allowing low-quality providers to operate indefinitely. The inspector who misses critical issues faces no market consequences. The agent who provides terrible service maintains equal standing with excellence. Without persistent performance data, every transaction becomes a gamble on provider quality. Bad actors don't just survive—they thrive in the shadows of information asymmetry, protected by the same opacity that enables systemic extraction.

