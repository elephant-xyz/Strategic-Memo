\section{Intermediaries and Gatekeepers}

The principal-agent problem in real estate has metastasized into an entire economic layer. Multiple Listing Services control property data with iron fists, charging thousands annually for information provided free by sellers. Government-Sponsored Enterprises mandate specific transaction pathways that maximize intermediary involvement regardless of consumer benefit. These gatekeepers don't facilitate transactions—they tax them. By controlling data access and enforcing narrow transaction corridors, they guarantee their position in every deal, extracting value through control rather than creation.

Manual systems compound this gatekeeping into Byzantine complexity. A typical transaction involves seventeen separate counterparties: buyer agents, seller agents, mortgage brokers, lenders, appraisers, inspectors, title agents, escrow officers, title insurers, and their various support staff. Each maintains partial, incompatible records. Each charges for their fragment of truth. None communicate effectively with others. The absence of interoperability isn't a technical limitation—it's a business model. Fragmentation ensures repeated work, repeated fees, and repeated opportunities for extraction.

The human cost extends beyond money to time, stress, and lost opportunities. Families spend months navigating this labyrinth, collecting documents that were collected before, verifying information that was already verified, paying for services that add no value beyond navigating the complexity that the system itself creates. If you buy a house today and attempt to sell it tomorrow, you start from zero—every verification, every inspection, every approval must be repeated as if the property had just materialized from quantum foam.

