\section{Integration with Protocol Architecture}

Bitstrapping integrates seamlessly with protocol design without compromising core functionality. The funding mechanism exists as a separate layer, interacting with but not dependent on protocol operations. This separation ensures fundraising doesn't distort technical architecture or governance structures—a critical flaw in many token-based systems.

For Elephant specifically, the integration follows clear boundaries. Bitcoin holders lock BTC in a dedicated smart contract for 3 years, receiving MAHOUT tokens from the DAO treasury allocation at a 30\% discount to projected utility value. Since MAHOUT tokens are primarily mined by oracles through Proof of Truth, the DAO allocates a portion of its 30\% treasury share for Bitstrapping participants. Elephant commits to repay 15\% of locked Bitcoin value in stablecoins at maturity. MAHOUT tokens provide concrete utility: 50\% transaction fee discounts, governance rights proportional to holdings, priority oracle assignment access, and shares of protocol revenue exceeding operational costs.

The smart contract architecture ensures trustless execution through cryptographic commitments:

\begin{itemize}
\item Time-locked Bitcoin remains verifiable on-chain, providing transparency and preventing protocol malfeasance
\item Token minting follows predetermined curves based on Bitcoin contributions, preventing dilution or favoritism
\item Repayment obligations are hard-coded and auditable, creating credible commitments
\item Governance remains separate from fundraising, preventing capture by large Bitcoin contributors
\end{itemize}

Integration points remain minimal and explicit. The Bitstrapping contract interacts with the protocol only to mint tokens according to contribution rules. It doesn't affect oracle operations, data verification processes, or service provider dynamics. This modularity ensures the funding mechanism enhances rather than complicates protocol development.

Long-term sustainability emerges from aligned incentives rather than enforced rules. Bitcoin contributors want the protocol to succeed to maximize token utility value. The protocol wants to succeed to easily meet repayment obligations from revenues. Users benefit from better-funded infrastructure. Everyone wins when the protocol creates real value—a stark contrast to the zero-sum games of traditional venture funding.

