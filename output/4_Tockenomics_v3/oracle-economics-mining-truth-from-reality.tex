\section{Oracle Economics: Mining Truth from Reality}

Oracles form the backbone of Elephant's economy, transforming real-world property data into blockchain-verified truth. The protocol divides each property into 20 independent fact groups—ownership, mortgages, liens, permits, valuations, and others—each requiring verification from three independent oracles who must reach unanimous consensus. This granular approach prevents any single entity from controlling property records while ensuring thorough verification.

The reward structure creates urgency without sacrificing accuracy. For each fact group successfully verified, exactly 1 MAHOUT token is minted and distributed: approximately 81\% to the first oracle, 16\% to the second, and 3\% to the third. This exponential curve incentivizes rapid response while still rewarding confirmation. An oracle spotting a property sale can race to verify ownership changes, earning 0.81 MAHOUT if first, while validators earn smaller but still meaningful rewards.

\[
\text{Oracle Rewards per Fact Group} = \begin{cases}
\text{First: } 0.81 \text{ MAHOUT} \\
\text{Second: } 0.16 \text{ MAHOUT} \\
\text{Third: } 0.03 \text{ MAHOUT} \\
\text{Total: } 1.00 \text{ MAHOUT}
\end{cases}
\]

But initial verification is just the beginning. Oracles must maintain their assigned fact groups in real-time, updating changes as they occur. If a property sells, gets renovated, or faces liens, the responsible oracle must submit updates immediately. Failure to maintain current data doesn't just forfeit future rewards—it strips vMAHOUT governance tokens from negligent oracles, transferring them to whoever submits the update. This creates an ecosystem of vigilant maintenance where data quality continuously improves.

Oracle participation requires staking MAHOUT tokens, with higher stakes earning priority access to unverified properties entering the system. This stake faces slashing penalties for submitting false data, creating skin in the game that ensures dedication to accuracy. The economic model transforms data verification from a cost center—currently consuming \$8,600 per transaction—into a profit center for diligent oracles.

