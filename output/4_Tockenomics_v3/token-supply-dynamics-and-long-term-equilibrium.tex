\section{Token Supply Dynamics and Long-Term Equilibrium}

MAHOUT supply grows deterministically with real-world property verification rather than arbitrary emission schedules. With approximately 140 million properties in the United States and 20 fact groups per property, the maximum theoretical supply reaches 2.8 billion MAHOUT tokens. However, practical supply remains much lower as not all properties require verification simultaneously and some fact groups rarely change.

Initial supply growth follows an S-curve: slow initial adoption as early oracles verify high-value properties, rapid expansion as the network proves its value, then gradual deceleration as most properties join the system. We project 50 million tokens minted in Year 1, 200 million by Year 3, and approaching 500 million by Year 5 as the system reaches maturity.

Long-term equilibrium emerges from balanced supply and demand. Oracle rewards for new verifications decrease as fewer unverified properties remain, while update rewards for maintaining data quality continue indefinitely. Transaction fees create continuous token demand while staking for advertisements locks supply. The burn mechanism from governance transfers adds deflationary pressure. These forces create a sustainable economy where token value reflects network utility rather than speculation.

The total addressable market extends far beyond initial US deployment. International expansion multiplies potential supply while creating new demand centers. Each country's property records require verification, each market needs service providers, each transaction generates fees. The token economy scales globally while maintaining local relevance through property-specific fact sheets and service provider competition.

This economic model achieves what current real estate systems cannot: aligned incentives where every participant profits from improving the system rather than extracting from it. Oracles profit from truth. Service providers profit from quality. Users profit from transparency. The protocol profits from growth. Value flows to contributors rather than gatekeepers, creating sustainable economics for the largest asset class in human history.