\section{Service Provider Economics: Competing on Merit}

Real estate professionals face a fundamental shift in the Elephant economy: from gatekeeping to genuine service competition. The protocol automatically generates SEO-optimized fact sheets for every verified property, designed to rank highly in search results and attract millions of property researchers. These fact sheets become premium advertising real estate where professionals compete for visibility.

Service providers—agents, lenders, inspectors, attorneys—stake MAHOUT tokens to secure advertising positions on relevant fact sheets. Higher stakes win more prominent placement, creating a pure market for attention. But unlike traditional advertising where money alone determines position, the Elephant protocol incorporates performance metrics. Service providers who generate positive outcomes maintain their positions with lower stakes, while those with poor reviews require increasingly higher stakes to remain visible.

This staking mechanism generates continuous demand for MAHOUT tokens while funding the oracles who maintain the underlying data. A portion of all staking fees flows back to the oracles responsible for each property's fact groups, creating recursive incentives for data quality. Better data attracts more users, driving more advertising value, increasing staking competition, rewarding better oracles, who create better data.

The economics fundamentally restructure professional incentives. Currently, agents optimize for transaction volume and commission maximization. In the Elephant economy, they optimize for client satisfaction and service quality, as positive outcomes reduce their advertising costs. This transforms marketing spend from pure expense to investment in reputation that compounds over time.

