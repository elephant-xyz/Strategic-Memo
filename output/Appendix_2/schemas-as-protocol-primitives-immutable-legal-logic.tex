\section{Schemas as Protocol Primitives: Immutable Legal Logic}

Schemas transcend mere data definitions to become protocol-level primitives that encode legal and business logic into immutable, verifiable specifications. This elevation of schemas from implementation details to first-class objects enables Elephant to adapt to any jurisdiction, asset type, or transaction pattern without protocol changes.

Each schema undergoes rigorous lifecycle management. Development begins with stakeholder input—lawyers, real estate professionals, technologists collaborate on requirements. The schema is drafted using Elephant's Schema Definition Language (SDL), which extends JSON Schema with legal semantics. Community review ensures completeness and compatibility. DAO governance approves final schemas through on-chain voting. Approved schemas are published to IPFS, creating permanent CIDs. The protocol registry maps semantic names to CIDs for discovery.

This process ensures schemas reflect real-world needs rather than theoretical completeness. A jurisdiction discovering unique requirements can propose new schemas without waiting for protocol updates. Market adoption naturally selects useful schemas while obsolete ones remain available but unused. This evolutionary process creates a living legal framework that adapts to changing requirements while maintaining historical compatibility.

Schema immutability provides critical legal certainty. A relationship created under a specific schema version remains valid indefinitely under those rules. Legal disputes reference the exact schema version used at transaction time. This creates unprecedented clarity—the rules governing any historical transaction are permanently accessible and unambiguous. Traditional legal systems struggle with changing laws and interpretations. Elephant schemas provide mathematical certainty about applicable rules.

