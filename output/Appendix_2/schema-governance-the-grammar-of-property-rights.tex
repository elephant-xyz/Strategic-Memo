\section{Schema Governance: The Grammar of Property Rights}

Schemas in Elephant function as the constitutional framework governing all data relationships. Unlike traditional database schemas that merely structure data, Elephant schemas encode legal semantics, validation rules, and cross-jurisdictional compatibility into immutable, version-controlled specifications that ensure global coherence while allowing local adaptation.

Each schema defines a Group—a semantic container for related interaction patterns. The "Ownership" group contains schemas for various ownership types: simple ownership, joint tenancy, tenancy in common, corporate ownership, trust ownership. Each schema within the group shares common interfaces while implementing specific legal logic:

\begin{itemize}
\item Entity type constraints (e.g., only Natural Persons can be joint tenants)
\item Relationship cardinality (e.g., property can have multiple owners but each share must sum to 100\%)
\item Validation predicates (e.g., trust ownership requires valid trust documentation)
\item Temporal constraints (e.g., life estates terminate upon grantor death)
\end{itemize}

Schema governance operates through decentralized consensus rather than centralized authority. New schemas are proposed, reviewed, and adopted through DAO governance. This ensures schemas reflect actual usage needs rather than theoretical completeness. Popular schemas gain network effects as more relationships reference them, creating natural standardization without enforcing rigid uniformity.

The versioning system enables evolution without breaking existing relationships. Schema v2 can extend v1 with additional fields or constraints while maintaining backward compatibility. Relationships explicitly reference schema versions, ensuring perpetual interpretability. A relationship created in 2025 remains verifiable in 2050 using its original schema, even as newer versions emerge.

This governance model solves the protocol ossification problem plaguing many blockchains. Bitcoin's simplicity ensures security but limits expressiveness. Ethereum's complexity enables innovation but complicates verification. Elephant's schema governance achieves both: simple base protocol with complex semantics layered through governed schemas. The protocol remains minimal while applications gain unlimited expressiveness.

