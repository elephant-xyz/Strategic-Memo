\section{Globalized Data Model: Jurisdiction-Agnostic Architecture}

Elephant's data model achieves true global applicability by making no assumptions about local legal systems while providing complete flexibility for jurisdiction-specific requirements. This neutrality enables a single protocol to handle property transactions from Manhattan skyscrapers to Mumbai apartments to São Paulo favelas without architectural changes.

The key insight: legal differences are schema variations, not protocol modifications. US property ownership differs from UK leasehold systems in schema definitions, not fundamental architecture. Islamic finance prohibitions on interest require different mortgage schemas, not different protocols. Community land trusts, indigenous property rights, and socialist property models become schema variations rather than incompatible systems.

This flexibility extends to evolution within jurisdictions. When regulations change, new schema versions capture new requirements while historical transactions remain valid under original schemas. Grandfathering provisions are explicit in schema version references rather than complex legal interpretations. Cross-border transactions reference multiple schemas, ensuring compliance with all relevant jurisdictions.

The protocol provides schema composition primitives for complex scenarios. Multi-jurisdictional transactions compose schemas: a Canadian buying US property through a Cayman Islands entity references Canadian, US, and Cayman schemas. Each schema validates its relevant aspects while composition rules ensure compatibility. This modular approach scales to arbitrary complexity without central coordination.

Dispute resolution benefits from explicit schema references. Rather than arguing about applicable law, parties reference specific schema versions. Smart contracts can encode choice-of-schema clauses similar to choice-of-law provisions. This clarity reduces legal uncertainty and enables automated dispute resolution for schema-defined issues.

