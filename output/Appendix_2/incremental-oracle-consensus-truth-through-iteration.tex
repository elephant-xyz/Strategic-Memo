\section{Incremental Oracle Consensus: Truth Through Iteration}

Oracle consensus in Elephant abandons the problematic requirement for simultaneous agreement, instead embracing incremental validation that mirrors how truth emerges in real-world property transactions. This design recognizes that perfect information rarely exists at transaction time—truth accumulates through progressive verification by independent parties.

The incremental model works through staged validation. Initial oracles might verify basic ownership documents, adding their signatures to create a baseline trust level. Subsequent oracles verify additional aspects: title search results, lien status, tax records, physical inspection reports. Each oracle adds their signature and specific attestations, building a Merkle tree of progressive validation:

\[
\text{Validation}_{\text{tree}} = \text{MerkleTree}([\text{Oracle}_1(\text{ownership}), \text{Oracle}_2(\text{liens}), \text{Oracle}_3(\text{taxes}), ...])
\]

This structure provides graduated trust levels. A relationship with three oracle signatures offers basic confidence. Ten signatures including specialized validators (title companies, attorneys, inspectors) provide high confidence. The market determines required validation levels for different transaction types—buying a \$100k property might require fewer validations than a \$10M commercial complex.

Incremental consensus solves the oracle availability problem. Rather than requiring all oracles to be online simultaneously, each can contribute asynchronously. This enables global oracle networks spanning time zones and jurisdictions. It also allows specialized oracles to focus on their expertise—title researchers validate ownership history, inspectors validate physical condition, attorneys validate legal compliance.

The Merkle structure ensures efficient verification regardless of validation depth. Proving basic ownership requires only the first few oracle signatures. Proving institutional-grade verification might require the complete validation tree. This selective verification enables privacy-preserving proofs—parties see only the validation level relevant to their risk tolerance.

