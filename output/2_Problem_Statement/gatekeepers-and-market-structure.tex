\section{Gatekeepers and Market Structure}

The real estate industry's gatekeeping structure extracts \$339.3 billion annually through controlled access rather than value creation. Every transaction layer imposes access fees justified by regulatory requirements or information control. MLS organizations charge thousands for data access. Title insurance companies collect \$4{,}000 for protection against their own record-keeping failures. Lenders add margins justified by "relationship value" that amounts to phone calls and emails.

Recent legal challenges, particularly the NAR settlement, highlight growing recognition that current structures cannot persist. Yet litigation addresses symptoms rather than architecture. Reducing visible commission percentages without eliminating commission financing merely shifts extraction from transparent to hidden mechanisms. True reform requires fundamental reconstruction of how property transactions occur, not marginal adjustments to existing extraction systems.

Market evolution creates unprecedented disruption opportunities for protocol solutions. Consumer awareness of transaction costs reaches all-time highs. Regulatory pressure mounts on traditional gatekeepers. Technology maturity enables alternatives previously impossible. The convergence of these forces opens a window for architectural transformation that eliminates gatekeeping through mathematical trust and automated execution. The question is not whether change will come, but who will build the infrastructure for property's next century.