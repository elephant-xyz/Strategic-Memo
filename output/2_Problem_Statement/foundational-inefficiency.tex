\section{Foundational Inefficiency}

Real estate stands as the largest yet most inefficient consumer transaction category in human history. Each property transaction imposes \$67{,}532 in friction costs—a staggering burden that represents not merely inefficiency but systematic value extraction through architectural dysfunction. For families buying and selling homes multiple times throughout their lives, these costs compound into hundreds of thousands in lifetime transaction fees.

The current infrastructure treats each transaction as if property data never existed before. Despite properties remaining fundamentally unchanged between sales, the system demands complete re-verification of every detail. This designed amnesia benefits intermediaries who profit from repeated work while imposing unnecessary costs on consumers. When transaction costs equal 2\% of property value annually over typical seven-year holding periods, the system effectively taxes homeownership at rates rivaling property taxes themselves.

Data accessibility exemplifies the dysfunction. Multiple Listing Services (MLS) operate as regional monopolies, charging thousands in access fees while providing technology inferior to free consumer platforms. These gatekeepers profit not from innovation but from regulatory capture—controlling access to data that should flow freely in competitive markets. The \$40{,}100 in labor costs per transaction reflects not the complexity of matching buyers with sellers, but the inefficiency of systems designed before digital communication existed.

