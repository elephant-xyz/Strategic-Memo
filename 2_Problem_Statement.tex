\chapter{Problem Statement}

The American real estate market operates through deliberate forgetting. Every property transaction begins at zero knowledge, requiring full re-verification of information that was verified last year, and the year before, and every year stretching back decades. This engineered memory loss costs billions per transaction in redundant data verification---money spent not to discover new information but to rediscover what was already known. Multiply this across 5.5 million annual transactions, and we burn \$31.1 billion yearly on collective forgetfulness, a tax that enriches verification providers while impoverishing families.

This systematic inefficiency represents merely the visible symptom of deeper structural failures. The real estate industry wasn't designed for efficiency or transparency---it was architected by intermediaries, for intermediaries. Agents, brokers, bureaucrats, and administrators, almost none of whom possess technical expertise, have constructed manual systems of staggering complexity. As regulation expands and data requirements multiply, these analog processes don't scale linearly but exponentially, creating ever-more-lucrative opportunities for gatekeeping. The solution is a real estate oracle network which removes gatekeepers, creates persistent data memory, and creates transparency of process and pricing.

\section{Centralized Control}

The names themselves clearly signal intermediary roles---broker, agent---with the inevitable principal-agent problem. These administrators, bureaucrats, brokers, and agents, all of whom are non-technical, have built systems that reflect their limitations. Manual processes become non-linearly complex and expensive to operate as both regulation and data requirements increase.

MLS and GSE gatekeepers control data and enforce mandated transaction pathways with limited consumer choice. By controlling access to property listings and transaction infrastructure, they guarantee their position in every deal. Multiple separate counterparties maintain partial, incompatible records. Each charges for their fragment of truth. None communicate effectively with others. The absence of interoperability isn't a technical limitation---it's a business model. Fragmentation ensures repeated work, repeated fees, and repeated opportunities for extraction.

However, the central authorities intentionally do not publish data about themselves---service provider quality, average fees, closed volumes---which allows low-quality vendors to remain in the market and service unwitting customers without any consequence. Given the protected position of the gatekeepers, there are no market forces to incentivize providing a better product at a lower cost.

Blocking open competition through licensing requirements, wasteful 99-hour continuing education mandates, ethics pledges enforced by the NAR, overtly signaled price collusion maintaining 6\% commissions, and heavy-handed competitive restrictions as seen in battles between Rocket Mortgage and UWM---these mechanisms entrench the status quo. The system works exactly as designed, extracting maximum value while providing minimum service.

\section{Misaligned Incentives}

Intermediaries always introduce the principal-agent problem. Friction is always the core product of a gatekeeper. Administrators, bureaucrats, brokers, and agents design manual systems. Manual systems become non-linearly complex and non-linearly expensive to operate as both regulation and data requirements have increased monotonically over time, especially after 2008. This friction creates revenue for the gatekeepers.

Fragmentation and forgetting are also features---fragmented systems require repeated aggregation and reconciliation across numerous counterparties per transaction. Inefficient processes are mandated to be repeated from scratch across every transaction. Lack of data memory forces full re-verification of title, appraisal, inspection, lien, and servicing data on every transaction.

If you buy a house today and try to sell it tomorrow, you have to totally start over and pay for everything again. This should be trivially easy to do, yet the current system makes it impossibly expensive. No service providers have the incentive to provide a better product at a lower cost. The absence of reputation tracking allows repeated use of low-quality vendors and intermediaries without market penalties. Bad actors thrive in the shadows of information asymmetry, protected by the same opacity that enables systemic extraction.

\section{Rent-Seeking}

Agent and broker fees are tied to the house price---a fundamental misalignment where compensation scales with asset value instead of work performed. Fees are expertly designed to feel less painful since the seller pays both the buy and sell side commission, which eliminates any pain the buyer feels. The seller pays real estate commissions off the top, which minimizes the pain because they never have to actually send a wire for the true amount.

The result is that real estate agent fees are 2x other developed countries as a percentage of house price. Real estate agents tell the consumer that all of these transaction costs are normal. Lending structures embed commissions inside the interest rate via pricing spreads that inflate borrowing costs and cause excess interest. The result is that excess interest is the single largest transaction cost but the one that is the most expertly concealed.

The real estate agent jealously guards the relationship with the consumer. The real estate agent therefore controls selection of downstream service providers. These service providers are selected based on loyalty and relationship history, not price and service quality. For discerning consumers who inspect the Closing Disclosure, suddenly thousands of dollars of service provider fees (inspection, appraisal, credit, title) suddenly seem reasonable in comparison to the massive commissions that have already exhausted the customer.

A borrower comparing mortgage rates sees numbers like 6.5\% versus 7\%, not understanding that the difference represents tens of thousands in hidden commissions compounded over decades. The true cost structure remains deliberately opaque, ensuring consumers cannot make informed decisions about the services they purchase.

\section{Price Inflation}

The incentives of the real estate agent, mortgage broker, other service providers and seller are to inflate the asset price. This is a key contributor to housing unaffordability. Natural competitive market forces are totally absent since competition is stifled and the true price/cost of everything is expertly hidden and muted.

Fee layers accumulate across multiple refinancing cycles and hold periods, compounding consumer cost over time and creating high-water marks that keep prices going up. A family owning a home for seven years pays \$9,594 annually just in transaction costs---a hidden tax that enriches intermediaries while impoverishing households.

The macroeconomic impact ripples through society. When transaction costs consume 16.3\% of property value, labor mobility freezes. Workers can't afford to relocate for better opportunities. Families delay moves, living in suboptimal housing because transaction costs are prohibitive. Young buyers are priced out entirely, not by home values but by transaction friction. Wealth accumulation stalls as equity evaporates into fees.

\section{Decentralized Remedy}

Centralization is the problem. Decentralization is the solution. Real estate data must be anchored on decentralized blockchain rails as public infrastructure. Property records including ownership, mortgage, servicing, appraisal, and upgrades become independent data layers. Data is ingested through cryptographic attestations from independent providers. Staking, slashing, and rewards align data provider incentives.

Modular architecture allows verified data layers to attach incrementally. Borrower personal financial data remains off-chain in early protocol phases. Privacy-preserving cryptography will support borrower data in later phases. Industry memory captures longitudinal asset lifecycle and reputation history. Shared data access lowers verification costs, transaction friction, and capital market barriers.

This isn't incremental reform but systematic replacement. Where the current system profits from forgetting, we create permanent memory. Where gatekeepers control access, we enable permissionless participation. Where opacity enables extraction, we enforce transparency. Where misaligned incentives corrupt outcomes, we align rewards with value creation. The solution doesn't negotiate with the problem---it obsoletes it entirely.