\chapter{Problem Statement}

The American real estate market operates through deliberate forgetting. Every property transaction begins at zero knowledge, requiring full re-verification of information that was verified last year, and the year before, and every year stretching back decades. This engineered memory loss costs billions per transaction in redundant data verification—money spent not to discover new information but to rediscover what was already known. Multiply this across 5.5 million annual transactions, and we burn \$31.1 billion yearly on collective forgetfulness, a tax that enriches verification providers while impoverishing families.

This systematic inefficiency represents merely the visible symptom of deeper structural failures. The real estate industry wasn't designed for efficiency or transparency—it was architected by intermediaries, for intermediaries. Agents, brokers, bureaucrats, and administrators, almost none of whom possess technical expertise, have constructed manual systems of staggering complexity. As regulation expands and data requirements multiply, these analog processes don't scale linearly but exponentially, creating ever-more-lucrative opportunities for gatekeeping. The solution is a real estate oracle network which removes gatekeepers, creates persistent data memory, and creates transparency of process and pricing.

\section{Intermediaries and GateKeepers Control}

The names themselves clearly signal intermediary roles—broker, agent—with the inevitable principal-agent problem. These administrators, bureaucrats, brokers, and agents, all of whom are non-technical, have built systems that reflect their limitations. Manual processes become non-linearly complex and expensive to operate as both regulation and data requirements increase.

MLS and GSE gatekeepers control data and enforce mandated transaction pathways with limited consumer choice. By controlling access to property listings and transaction infrastructure, they guarantee their position in every deal. A typical transaction involves seventeen separate counterparties, each maintaining partial, incompatible records. Each charges for their fragment of truth. None communicate effectively with others. The absence of interoperability isn't a technical limitation—it's a business model. Fragmentation ensures repeated work, repeated fees, and repeated opportunities for extraction.

Blocking open competition through licensing requirements, wasteful 99-hour continuing education mandates, "ethics" pledges enforced by the NAR, overtly signaled price collusion maintaining 6\% commissions, and overtly signaled competitive restriction as seen in battles between Rocket Mortgage and UWM—these mechanisms entrench the status quo. The system works exactly as designed, extracting maximum value while providing minimum service.

\section{Misaligned Incentives and Principal-Agent}

Service provider incentives are to design a system which forces repeated verification, redundant intermediaries, and inefficient processes across every transaction. Lack of data memory forces full re-verification of title, appraisal, inspection, lien, and servicing data on every transaction. Fragmented systems require repeated aggregation and reconciliation across approximately 17 counterparties per transaction.

If you buy a house today and try to sell it tomorrow, you have to totally start over and pay for everything again. This should be trivially easy to do, yet the current system makes it impossibly expensive. No service providers have the incentive to provide a better product at a lower cost. The absence of reputation tracking allows repeated use of low-quality vendors and intermediaries without market penalties. Bad actors thrive in the shadows of information asymmetry, protected by the same opacity that enables systemic extraction.

\section{Institutionalized Rent-Seeking}

Agent and broker fees are tied to the house price—a fundamental misalignment where compensation scales with asset value rather than work performed. Transaction fees for inspection, appraisal, credit, and title are dwarfed in comparison to house price, making them seem reasonable even when overpriced. 

Fees are expertly designed to feel less painful since sellers pay real estate commissions "off the top" and buyers unwittingly finance all mortgage commissions inside their rate. Intermediaries control key steps in agent commissions, broker fees, lender fees, and processing charges across every transaction. Lending structures embed hidden pricing spreads that inflate borrowing costs through rate markups tied to fees and leverage.

Real estate agent fees are 2x other developed countries as a percentage of house price. Lending commission and excess interest make up 40\% of transaction costs, yet remain largely invisible to consumers. A borrower comparing mortgage rates sees numbers like 6.5\% versus 7\%, not understanding that the difference represents tens of thousands in hidden commissions compounded over decades.

\section{Resulting Housing Unaffordability}

Natural competitive market forces are totally absent since competition is stifled and the true price/cost of everything is expertly hidden and muted. Fee layers accumulate across multiple refinancing cycles and hold periods, compounding consumer cost over time. A family owning a home for seven years pays \$9,594 annually just in transaction costs—a hidden tax that enriches intermediaries while impoverishing households.

The macroeconomic impact ripples through society. When transaction costs consume 16.3\% of property value, labor mobility freezes. Workers can't afford to relocate for better opportunities. Families delay moves, living in suboptimal housing because transaction costs are prohibitive. Young buyers are priced out entirely, not by home values but by transaction friction. Wealth accumulation stalls as equity evaporates into fees. The entire economy suffers from misallocated resources, reduced productivity, and decreased dynamism.

\section{Solution Principles}

Centralization is the problem. Decentralization is the solution. Real estate data must be anchored on decentralized blockchain rails as public infrastructure. Property records including ownership, mortgage, servicing, appraisal, and upgrades become independent data layers. Data is ingested through cryptographic attestations from independent providers. Staking, slashing, and rewards align data provider incentives. Modular architecture allows verified data layers to attach incrementally.

Borrower personal financial data remains off-chain in early protocol phases. Privacy-preserving cryptography will support borrower data in later phases. Industry memory captures longitudinal asset lifecycle and reputation history. Shared data access lowers verification costs, transaction friction, and capital market barriers.

This isn't incremental reform but systematic replacement. Where the current system profits from forgetting, we create permanent memory. Where gatekeepers control access, we enable permissionless participation. Where opacity enables extraction, we enforce transparency. Where misaligned incentives corrupt outcomes, we align rewards with value creation. The solution doesn't negotiate with the problem—it obsoletes it entirely.